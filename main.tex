\documentclass[12pt]{article}
\usepackage[utf8]{inputenc}
\usepackage[tmargin=1in,lmargin=1in,rmargin=1in]{geometry}
\usepackage{fancyhdr}
\setcounter{tocdepth}{5}
\usepackage{titletoc,tocloft}
\setlength{\cftsubsecindent}{0.65cm}
\setlength{\cftsubsubsecindent}{1.3cm}
\newcommand{\nocontentsline}[3]{}
\let\origcontentsline\addcontentsline
\newcommand\stoptoc{\let\addcontentsline\nocontentsline}
\newcommand\resumetoc{\let\addcontentsline\origcontentsline}
\usepackage{multicol}
\usepackage{setspace}
\usepackage{graphicx} % Required for inserting images
\usepackage{etoolbox}
\usepackage{enumitem}
\setlist[itemize]{noitemsep}
\usepackage{listings} % For code listings
\usepackage{color} % For colored text
\usepackage{amsmath} % For mathematical notation
\usepackage{array} % For better table formatting
\usepackage{float}
\patchcmd{\thebibliography}{\section*}{\section}{}{}
\graphicspath{ {./images/} }

\newcommand{\quickimage}[2]{%
\begin{figure}[!htbp]
\centering
\includegraphics[width=#2]{#1}
\end{figure}%
} 

\newcommand{\quickfigure}[4]{%
\begin{figure}[!htbp]
\centering
\includegraphics[width=#2]{#1}
\caption{#3}
\label{#4}
\end{figure}%
} 

\lstset{ %
  numbers=left,           
  stepnumber=1,           
  numbersep=5pt,          
  showspaces=false,           
  showstringspaces=false,     
  showtabs=false,         
  frame=single,           
  tabsize=4,              
  captionpos=b,           
  breaklines=true,            
  breakatwhitespace=false,        
}

%%%%%%%%%%%%%%%%%%%%%%%%%%%%%%%%%%%%%%%%%%%%%%%%%%%%%%%%%%%%%%%%%%%%%%%%%%
%                           Title Page                                   %
%%%%%%%%%%%%%%%%%%%%%%%%%%%%%%%%%%%%%%%%%%%%%%%%%%%%%%%%%%%%%%%%%%%%%%%%%%
\begin{document}

\begin{titlepage}
    \begin{center}
        \raisebox{-0.5\height}{\includegraphics[width=4cm]{template/Picture2.jpg}} \hfill
        \raisebox{-0.5\height}{\includegraphics[width=6cm]{template/Picture1.png}} \hfill
        \raisebox{-0.5\height}{\includegraphics[width=4cm]{template/pki-logo.png}}
        
        {\Large College of Engineering\\
        Electrical and Computer Engineering Department\\}
        
        \vspace{2cm}
        {\Large ECEN 495}
        \vspace{0.5cm}
        
        {\Huge The Harm Drone}
        \vspace{3cm}
        
        {\large By\\}
        \vspace{0.1cm}
        \begin{Large}
        Carter Brehm \\
        \vspace{0.1cm}
        Toby Heinemann \\
        \vspace{0.1cm}
        Camryn Klintworth \\
        \vspace{0.1cm}
        Michael Maline\\
        \end{Large}
        
        \vspace{3cm}
        
        \small Submitted in Partial Fulfillment of the Requirements for the B.Sc. Degree,\\
        \small Electrical and Computer Engineering,\\
        \small College of Engineering,\\
        \small University of Nebraska-Lincoln,\\
        \small Peter Kiewit Institute, Omaha, Nebraska, U.S.A.\\
        \small May 2025\\
    \end{center}
    \thispagestyle{empty}
\end{titlepage}

\newpage

%%%%%%%%%%%%%%%%%%%%%%%%%%%%%%%%%%%%%%%%%%%%%%%%%%%%%%%%%%%%%%%%%%%%%%%%%%
%                        Front Matter                                    %
%%%%%%%%%%%%%%%%%%%%%%%%%%%%%%%%%%%%%%%%%%%%%%%%%%%%%%%%%%%%%%%%%%%%%%%%%%
\section*{Front Matter}

\subsection*{Ethical Design Statement}
A statement affirming that the Code of Ethics of the IEEE has been reviewed and applied to the design process and in the selection of the final proposed design. And that, the designers have held the safety of the public to be paramount and have addressed this in their design.

\subsection*{Environmental Impact Statement}
A statement declaring that the designers have considered any impact to the environment in their design and have eliminated the use of toxic materials, specifically those dangerous to humans.

\subsection*{Project Abstract}
% (Insert Project Abstract as stipulated in the rubric)

\subsection*{Acknowledgement}
Group 127 would like to acknowledge the invaluable support and assistance provided by faculty members of the University of Nebraska.
\paragraph{Dr. Herbert Harms}'s enthusiasm for drone-based research and data collection was what inspired Group 127 to begin the project, and his continued financial support allowed us to deliver a completed product.
\paragraph{Dr. Hamid Sharif}'s lessons on PCB design, architecture, and more specifically, USB communication, were invaluable to the completion of the project.
\paragraph{Mr. Detloff}'s instruction regarding working as a multi-diciplinary, high-performance engineering team helped us to effectively distribute work, complete milestones, and finalize the project effectively.

%%%%%%%%%%%%%%%%%%%%%%%%%%%%%%%%%%%%%%%%%%%%%%%%%%%%%%%%%%%%%%%%%%%%%%%%%%
%                     Executive Summary                                  %
%%%%%%%%%%%%%%%%%%%%%%%%%%%%%%%%%%%%%%%%%%%%%%%%%%%%%%%%%%%%%%%%%%%%%%%%%%
\section*{Executive Summary (1 page)}

\subsection*{Foreword}
\begin{itemize}[noitemsep]
    \item Give the problem statement including the organizational problem, (the purpose of capstone projects, the context of your project) and the general technical problem (the type of project you are doing (software prototype, hardware prototype, application program for a client, etc.).
    \item Give a more specific assignment statement --- specifically what the writers of the report were asked to do (an overview of the project goals), the technical questions, task, and perhaps the hypothesis or solution.
    \item State the overall purpose of the report.
\end{itemize}

\subsection*{Summary}
\begin{itemize}[noitemsep]
    \item Provide the objective and background (how problem was approached, what were the results) including objective or hypothesis, methodology or experimental procedure and results.
    \item Give overall conclusions about the project including recommendations for improvements and their implications, subsequent action, and cost and benefits.
\end{itemize}

%%%%%%%%%%%%%%%%%%%%%%%%%%%%%%%%%%%%%%%%%%%%%%%%%%%%%%%%%%%%%%%%%%%%%%%%%%
%                     Table of Contents                                  %
%%%%%%%%%%%%%%%%%%%%%%%%%%%%%%%%%%%%%%%%%%%%%%%%%%%%%%%%%%%%%%%%%%%%%%%%%%
\tableofcontents
\newpage

%%%%%%%%%%%%%%%%%%%%%%%%%%%%%%%%%%%%%%%%%%%%%%%%%%%%%%%%%%%%%%%%%%%%%%%%%%
%                         Introduction                                   %
%%%%%%%%%%%%%%%%%%%%%%%%%%%%%%%%%%%%%%%%%%%%%%%%%%%%%%%%%%%%%%%%%%%%%%%%%%
\section{Introduction (3-6 pages)}
The introduction should orient the reader to the topic of the report by including the following:
\begin{itemize}[noitemsep]
    \item \textbf{The problem} - Explain the problem that is addressed in the report.
    \item \textbf{The objective} - State the objectives (what your project needs to accomplish to solve the problem).
    \item \textbf{The method of the report} - Describe the organization and structure of the report.
\end{itemize}

%%%%%%%%%%%%%%%%%%%%%%%%%%%%%%%%%%%%%%%%%%%%%%%%%%%%%%%%%%%%%%%%%%%%%%%%%%
%                    Problem Formulation                               %
%%%%%%%%%%%%%%%%%%%%%%%%%%%%%%%%%%%%%%%%%%%%%%%%%%%%%%%%%%%%%%%%%%%%%%%%%%
\section{Problem Formulation (3-6 pages)}
\subsection{Problem Statement}
%State the problem to be solved as indicated by the need (the university, industry sponsor, or self-proposed). Present the objectives and expectations of the need and constraints given to the problem.
In many academic and research applications, there is a growing demand for reliable and cost-effective unmanned aerial systems (UAS) capable of collecting and storing high-fidelity environmental or system data. However, current drone systems used in industry and academia frequently rely on expensive proprietary sensors, specialized storage modules, and wireless transmission protocols that are not only costly but also introduce reliability risks in field conditions. The result is a significant barrier to entry for budget-conscious research teams, student projects, and field operations requiring autonomy, robustness, and open-platform support.

The problem, therefore, is to design a universal, low-cost data acquisition and logging system for drones that supports modular USB sensor integration, onboard data storage with integrity verification, and compliance with operational safety standards. The solution must not rely on wireless data transmission, must store data in a platform-agnostic format, and must be rugged enough for field deployment in moderate outdoor environments.

This project is self-proposed and sponsored by the University of Nebraska Omaha in partial fulfillment of capstone design requirements. The expectations include delivering a working prototype that meets five measurable criteria (PSSCs), including USB sensor interfacing, SD card compatibility, nonvolatile memory recovery, IEEE 1789-compliant lighting, and successful cross-platform data retrieval. Constraints include a fixed budget of \$325, a maximum PCB size compatible with most hobby-grade drones (100~mm × 100~mm), and delivery by the designated final demonstration date.

\subsection{Background}
%Discuss the state of the art, competitive products, and patent liability analysis (HW 1).

Modern UAV applications span a wide range of fields, from agriculture and precision mapping to environmental monitoring, disaster response, and research support. In each case, data collected from sensors onboard the UAV is critical to the mission's success. Currently, many commercial UAVs use integrated, non-modular systems or require high-end payloads with proprietary interfaces (e.g., DJI’s XT2 or Micasense multispectral sensors), which cost hundreds to thousands of dollars. These products often include limited options for local data storage, requiring operators to rely on wireless links such as Wi-Fi, LTE, or proprietary telemetry protocols. While effective under certain conditions, these links can become unstable or entirely unusable in remote environments or electromagnetically noisy regions.

Academic UAV platforms—such as ArduPilot-based systems, Raspberry Pi drones, or Pixhawk-integrated quadcopters—do allow for open-source development and hardware extension. However, most lack standardized USB sensor interfacing, relying instead on I2C, SPI, or UART protocols. Integrating external sensors often requires deep firmware changes and power conversion boards, further increasing complexity and cost. Moreover, these systems rarely include robust data integrity checking, making them vulnerable to power loss or in-flight resets.

In terms of lighting systems, few small UAVs incorporate flicker-safe lighting. IEEE Standard 1789-2015 was developed to address biological risks associated with pulse-width modulated (PWM) light sources—especially those visible to human observers or operating in environments where safety and compliance are essential. While this standard is widely used in lighting and display engineering, it is largely ignored in hobby and research drone lighting systems due to added cost and design complexity.
As these challenges and gaps in current UAV systems became evident, it was necessary to explore the existing commercial landscape and intellectual property space to understand where our system could differentiate itself—both technologically and legally. A formal review of patents and competitive products was conducted to identify potential overlaps, draw inspiration from existing solutions, and ensure that \textit{The Harm Drone} does not infringe upon established claims or technologies.

\subsubsection{Results of Patent and Product Search}

To assess potential patent infringement risks for \textit{The Harm Drone}, it is essential to examine existing patents with similar functions. One of the most relevant is \textbf{US 11,501,483 B2}, \textit{Removable Sensor Payload System for Unmanned Aerial Vehicle Performing Media Capture and Property Analysis}, filed on November 15, 2022 \cite{patent1}. This patent describes a modular payload system that allows UAVs to swap sensors for media capture and automated aerial data collection. Given \textit{The Harm Drone}’s use of specialized sensors for autonomous data collection, there is a potential overlap, particularly in sensor payload design and data acquisition methods. To mitigate risks, \textit{The Harm Drone} must ensure its sensor integration does not replicate this removable payload system.

Another relevant patent is \textbf{US 2022/0404273 A1}, \textit{High-Altitude Airborne Remote Sensing}, published on December 22, 2022 \cite{patent2}. This patent describes high-altitude UAVs equipped with specialized sensors for environmental monitoring, where collected data is processed and transmitted to a ground station. \textit{The Harm Drone} operates at lower altitudes, but it shares key functionalities such as autonomous navigation, environmental data collection, and onboard processing. The risk of infringement exists if \textit{The Harm Drone}’s sensor integration and data handling closely resemble those in this patent.

Finally, \textbf{US 10,462,366 B1}, \textit{Autonomous Drone with Image Sensor}, granted on October 29, 2019 \cite{patent3}, describes a drone-based monitoring system that continuously collects image data, even while charging. It integrates stationary ground sensors with drone-captured data for real-time property surveillance and analysis. Key claims include image sensors for monitoring, data collection while docked, and integration with stationary sensors. \textit{The Harm Drone} also incorporates image sensors for environmental monitoring, posing a potential overlap in data collection, processing, and storage. However, \textit{The Harm Drone} does not perform continuous image capture while stationary, reducing the likelihood of direct infringement.

By differentiating its sensor integration, flight altitude, and data processing methods, \textit{The Harm Drone} can mitigate potential patent risks while maintaining its research-focused functionality.

\subsubsection{Analysis of Patent Liability}

To assess \textit{The Harm Drone}'s potential for patent infringement, we examine both \textit{literal infringement} and infringement under the \textit{doctrine of equivalents} concerning US 11,501,483 B2 \cite{patent1}, US 2022/0404273 A1 \cite{patent2}, and US 10,462,366 B1 \cite{patent3}. While \textit{The Harm Drone} shares functional similarities with these patents, significant differences in design, operation, and data collection methods reduce the likelihood of infringement.

\begin{itemize}
    \item \textbf{US 11,501,483 B2}: Describes a removable sensor payload system for UAVs, allowing modular sensor integration for media capture and property analysis. While \textit{The Harm Drone} also integrates specialized sensors, its design does not use interchangeable payloads but instead features fixed sensors for research applications. Additionally, it focuses on environmental monitoring rather than media capture or property analysis. Under the doctrine of equivalents, despite both projects involving sensor-equipped UAVs, \textit{The Harm Drone} prioritizes weather resistance and research-driven sensor integration, reducing infringement risks.

    \item \textbf{US 2022/0404273 A1}: Covers high-altitude UAV remote sensing with drones designed for real-time environmental monitoring. \textit{The Harm Drone}, however, operates at low-to-mid altitudes and relies on offline data storage rather than live transmission. Under literal infringement analysis, it does not match the altitude-dependent and real-time features of this patent. Under the doctrine of equivalents, differences in mission profile and data handling further distinguish the project.

    \item \textbf{US 10,462,366 B1}: Details an autonomous drone that collects image data continuously, even while docked. While \textit{The Harm Drone} uses an image sensor, it does not support continuous monitoring or stationary data capture. Instead, it only captures data during pre-programmed flight paths. Under both literal infringement and the doctrine of equivalents, the project's scientific and non-surveillance objectives offer meaningful differentiation.
\end{itemize}

In conclusion, \textit{The Harm Drone} is unlikely to literally infringe any of these patents due to differences in operational design, sensor integration, and data processing. While conceptual similarities exist in UAV-based data collection, distinctions in implementation and objectives further reduce patent conflict risks.

\subsubsection{Action Recommended}

To further mitigate patent infringement risks, \textit{The Harm Drone} should prioritize structural and operational differentiation rather than simply adding new features. Key recommendations include:

\begin{itemize}
    \item \textbf{Sensor Integration:} Avoid modular or swappable sensor payloads as described in US 11,501,483 B2. Fixed or semi-permanent sensor mounting and hardwired connections are preferred. Custom connectors rather than universal mounts will further distinguish the system.
    
    \item \textbf{Altitude and Transmission Constraints:} Operate strictly within low-to-mid altitude ranges and avoid any real-time data transmission to differentiate from US 2022/0404273 A1. Offline storage and post-flight data retrieval should be emphasized.

    \item \textbf{Image Sensor Limitations:} Ensure the image sensor captures data only during flight, at pre-programmed GPS waypoints. Avoid any stationary or docked data collection, in contrast to the system described in US 10,462,366 B1.

    \item \textbf{Custom Software and Algorithms:} Develop custom navigation and flight control software, rather than depending on proprietary or lightly modified open-source frameworks. This avoids infringing on patented control algorithms or autonomous behaviors.

    \item \textbf{Research Positioning:} Clearly brand and document \textit{The Harm Drone} as a tool for environmental research or agricultural data collection. Avoid surveillance, property monitoring, or media capture use cases to limit overlaps with broad UAV surveillance patents.
\end{itemize}

By implementing these strategic modifications—including non-removable sensor integration, altitude and data handling restrictions, waypoint-bound image capture, and unique flight software—\textit{The Harm Drone} will remain functionally distinct from the referenced patents. These actions will protect the project from infringement risk while achieving its design goals.
With a clear understanding of existing patent landscapes and actionable steps to avoid infringement, the focus of the project now shifts from legal differentiation to technical implementation. These patent-driven constraints directly inform the design problem, shaping both the hardware and software requirements. In the following section, the core engineering challenge is formally defined, incorporating key limitations such as non-removable sensors, offline operation, and standards compliance. This formulation ensures that the project not only avoids legal conflicts but also addresses real-world needs for robust, low-cost UAV data systems.

\subsection{Problem Formulation}
%Show that the problem has been formulated by presenting appropriate design method objectives, taking into consideration the following factors:
%\begin{itemize}[noitemsep]
   % \item (a) Problem is realistic or satisfies a specified need
   % \item (b) Easy to verify and/or validate by the end of the project
%\end{itemize}

To address the limitations of current UAV data systems, this project formulates a hardware and software co-design problem centered around developing a cost-effective, modular, and platform-independent onboard computing solution. The core engineering challenge is to design a drone-compatible embedded system capable of collecting sensor data from multiple sources, storing that data in a portable format, and maintaining operational reliability under autonomous, offline conditions—all while staying within a limited budget and adhering to safety standards.

This problem is realistic and addresses a well-defined need within both academic and applied research domains. Many existing UAV platforms used in environmental monitoring, fieldwork, or educational applications lack modular sensor integration, support only proprietary or complex communication protocols, and often rely on unreliable wireless transmission methods. These limitations significantly restrict accessibility for small teams, educators, and research groups working with limited financial and technical resources.

The formulated design problem requires the development of a compact embedded system that:
\begin{itemize}
    \item Communicates with multiple external sensors using standard, low-cost interfaces;
    \item Collects and stores high-resolution environmental data during flight;
    \item Maintains data integrity in the event of unexpected power loss;
    \item Stores collected data in a format readable across all major operating systems;
    \item Integrates a safety-compliant lighting system suitable for use around people;
    \item Operates without reliance on a wireless connection or cloud-based infrastructure;
    \item Remains compact, lightweight, and compatible with common UAV platforms.
\end{itemize}

The design method is structured to satisfy both engineering feasibility and verification at each stage of development. The system architecture must remain adaptable for testing under simulated or real UAV flight conditions, allowing for validation of each subsystem independently—such as power management, data handling, interface timing, and sensor communication. Design constraints related to size, power consumption, and data compatibility are selected specifically to ensure that the final product can be verified through bench testing, field simulation, and cross-platform data access.

This problem formulation enables an iterative design approach, incorporating hardware prototyping, embedded firmware development, and system-level validation. By focusing on modularity, offline data handling, and cross-platform accessibility, the project remains grounded in realistic engineering constraints while offering meaningful impact in real-world applications such as environmental monitoring, agriculture, and research support.
To translate this problem formulation into an actionable engineering plan, a detailed set of design requirements and specifications must be established. These specifications form the foundation for evaluating the success of the final system and ensure that all subsystems align with both project goals and external constraints, such as compliance standards and operational environments. By defining clear, quantifiable metrics, the design team can track progress, validate subsystem performance, and verify that the integrated system meets all functional and safety expectations. The following section outlines these specifications, along with the criteria that will be used to determine whether the project successfully achieves its technical objectives.

%%%%%%%%%%%%%%%%%%%%%%%%%%%%%%%%%%%%%%%%%%%%%%%%%%%%%%%%%%%%%%%%%%%%%%%%%%
%    Project Design Requirements, Specifications, and Success Criteria     %
%%%%%%%%%%%%%%%%%%%%%%%%%%%%%%%%%%%%%%%%%%%%%%%%%%%%%%%%%%%%%%%%%%%%%%%%%%
\section{Project Design Requirements, Specifications, and Success Criteria (3-6 pages)}
%\begin{itemize}[noitemsep]
  %  \item Give a clear set of design specifications for the project. The design specifications should be clear concise statements with a specific metric and an appropriate value.
  %  \item The specifications should provide an unambiguous measure of the success of the final design in meeting the need and constraints associated with the design problem.
  %  \item Problem Requirements Specifications is a dynamic process. Although it is desirable to freeze a set of requirements permanently, it is rarely possible. Requirements are likely to evolve through an iterative process that involves communication between the customer specifying the need and the technical community. The impact of proposed requirements must be evaluated to ensure that the initial intent of the requirements baseline is maintained or that changes to the intent are understood and accepted by the customer. The following diagram summarizes the dynamic process of developing system requirements:
%\end{itemize}

%\begin{center}
%Need or Customer \\
%Developed System Requirements \\
%Design Team Environment/Constraints \\

\begin{table}[H]
\centering
\begin{tabular}{|c|p{4cm}|p{3cm}|p{3cm}|p{4cm}|}
\hline
\textbf{Spec ID} & \textbf{Description} & \textbf{Metric} & \textbf{Target Value} & \textbf{Rationale} \\
\hline
DS1 & Onboard data retention & Data capacity & $\geq$ 1000 KB & Ensures retention of flight, log, and sensor data in persistent memory. \\
\hline
DS2 & Data accessibility & Cross-platform compatibility & Readable on Mac OS, Windows, Debian Linux & Guarantees log access on all major OS platforms. \\
\hline
DS3 & External storage capacity & SD card storage & $\geq$ 4 GB & Provides space for extended data logging. \\
\hline
DS4 & USB sensor support & Concurrent USB connections & Up to 4 sensors & Supports flexible external sensing integration. \\
\hline
DS5 & Light system compliance & Flicker modulation testing & Pass IEEE 1789-2015 check & Meets industry lighting safety standards. \\
\hline
DS6 & PCB fabrication & Trace and via tolerance & $\leq \pm$5 mil & Ensures manufacturability and reliability. \\
\hline
DS7 & USB data throughput & Data transfer rate & $\geq$ 12 Mbps & Enables real-time sensor data exchange. \\
\hline
DS8 & Firmware robustness & Uptime with full load & $\geq$ 8 hours & Confirms system stability. \\
\hline
DS9 & Boot time & Power-on to readiness & $\leq$ 5 seconds & Improves responsiveness. \\
\hline
DS10 & System packaging & Enclosure volume & $\leq$ 900 cm$^3$ & Ensures drone compatibility. \\
\hline
\end{tabular}
\caption{Project Design Specifications for Harm Drone System}
\end{table}

\subsection{Project Common Success Criteria}

The design team established the following success benchmarks:
\begin{enumerate}
    \item \textbf{Complete Bill of Materials:} A detailed and ordered BoM including all parts and vendors.
    \item \textbf{Design Schematic:} Readable schematic with complete interface and timing analysis.
    \item \textbf{PCB Fabrication:} Design, layout, and etching of a fully-functional custom PCB.
    \item \textbf{Board Assembly:} Populated board with debugged, verified electrical and logical operation.
    \item \textbf{Product Packaging:} Professionally finished enclosure demonstrating complete system functionality.
\end{enumerate}

These criteria serve as checkpoints throughout the project to ensure that major milestones are achieved before moving on to the next stage. They reflect not only technical success but also the team’s ability to manage procurement, documentation, debugging, and final presentation. These are non-negotiable goals that collectively define a complete and professional engineering process.

\subsection{Project-Specific Success Criteria (PSSC)}

\begin{enumerate}

    \item \textbf The board must receive and store 1000KB of flight, log, and sensor data on nonvolatile, persistent onboard storage.
    \item \textbf The SD card content must be accessible via Mac OS, Windows, and Debian Linux.
    \item \textbf The system must support bidirectional communication with up to 4 USB sensors.
    \item \textbf The drone's SD card must store up to 4GB of data, fully retrievable.
    \item \textbf The drone’s lighting must comply with IEEE Standard 1789-2015 guidelines.
\end{enumerate}

The Project-Specific Success Criteria (PSSCs) are the primary metrics used to validate whether the final design fulfills the essential functional expectations. Each PSSC represents a critical operational feature of the Harm Drone system and was developed in direct response to customer and course-defined requirements. These criteria will be assessed through experimental tests designed to simulate real-world operational environments. 

Together with the Project Common Success Criteria (PCSCs), the PSSCs form a comprehensive framework that balances engineering rigor, practical functionality, and user accessibility. Meeting both sets of criteria will demonstrate not only that the system works but that it is reliable, user-friendly, and designed according to real-world constraints and standards.

The PCSCs are especially valuable as they reflect the full life cycle of the hardware development process, from component selection and schematic design to layout, fabrication, and final integration. By breaking down the work into discrete engineering deliverables, the team ensures traceability and accountability at each stage. Each item in the PCSC list has a measurable output, such as a complete BoM or a fabricated PCB, which allows the team to gauge progress and verify quality throughout the project timeline.

Similarly, the PSSCs extend this concept by validating the functional performance of the final system. Where PCSCs capture engineering process quality, the PSSCs serve to evaluate whether the final design meets user needs, technical feasibility, and interoperability standards. By explicitly requiring things like cross-platform SD card access or USB sensor communication, the PSSCs reinforce user-oriented design and demonstrate that the team can meet practical, hands-on criteria for usability and effectiveness.
 that balances engineering rigor, practical functionality, and user accessibility. Meeting both sets of criteria will demonstrate not only that the system works but that it is reliable, user-friendly, and designed according to real-world constraints and standards.

\subsection{The Dynamic Nature of System Requirements}

System design requirements evolve iteratively through stakeholder interaction and technical discovery. The Harm Drone project acknowledges this reality with a dynamic process:

\begin{enumerate}
    \item \textbf{Initial Baseline Creation:} Requirements are derived from stakeholder needs, PSSCs, and technical constraints.
    \item \textbf{Customer-Engineer Dialogue:} Frequent communication ensures alignment and continuous feedback.
    \item \textbf{Impact Evaluation:} Any change undergoes technical, financial, and scheduling assessment.
    \item \textbf{Version Control:} Documented rationale and versioning ensure traceability and clarity.
\end{enumerate}

This dynamic model enables adaptability without compromising core functionality, encouraging a flexible yet controlled approach to engineering design.
The recognition that system requirements are not static, but rather evolve through iterative refinement, naturally leads into the conceptual development phase of the project. With a flexible yet well-documented baseline established, the design team was able to begin exploring multiple solution pathways that would satisfy both the technical objectives and operational constraints defined earlier. The following section outlines the conceptual design process, beginning with the literature that informed early decisions, and continuing through concept generation, reduction, and the formulation of a structured implementation plan.

%%%%%%%%%%%%%%%%%%%%%%%%%%%%%%%%%%%%%%%%%%%%%%%%%%%%%%%%%%%%%%%%%%%%%%%%%%
%         Concept Development, Synthesis, and Process Description          %
%%%%%%%%%%%%%%%%%%%%%%%%%%%%%%%%%%%%%%%%%%%%%%%%%%%%%%%%%%%%%%%%%%%%%%%%%%
\section{Concept Development, Synthesis, and Process Description (5-10 pages)}
\subsection{Literature Review}
Give a brief summary of the key literature that has been researched and used in the design effort. This can include textbooks, handbooks, technical papers, technical reports, web sources, codes and regulations. A summary of similar designs, processes, or techniques can also be discussed to show the strengths and weaknesses of your design compared to others. Indicate whenever the design process was supported by previous coursework.

\subsection{Concept Generation}
Show that design methods were used to generate several conceptual solutions to the design problem. Draw sketches or tree diagrams to describe the alternatives that were produced by this effort.

\subsection{Concept Reduction}
Show that a judicial decision-making process was used to reduce the number of possible conceptual solutions to a single (optimal) solution that is to be implemented and verified and/or validated by the end of the project. Discuss why alternative solutions were rejected/chosen over other solutions. Describe the criteria used to evaluate potential solutions. Substantiate that the proposed final concept is the optimal choice in providing the functionality necessary while best meeting the specified constraints of the design problem. Document in detail the decision-making process.

\subsection{Production Schedule (1-2 pages)}
Discuss the phases of the design and implementation of your project. (PERT charts, Gantt charts, or OPPMs may be appropriate in the discussion) Recommend any improvements that could have been made in the scheduling and planning.

%%%%%%%%%%%%%%%%%%%%%%%%%%%%%%%%%%%%%%%%%%%%%%%%%%%%%%%%%%%%%%%%%%%%%%%%%%
% Detailed Engineering Analysis and Design Product Presentation          %
%%%%%%%%%%%%%%%%%%%%%%%%%%%%%%%%%%%%%%%%%%%%%%%%%%%%%%%%%%%%%%%%%%%%%%%%%%
\section{Detailed Engineering Analysis and Design Product Presentation}
%Present and discuss the design concepts which have been used to solve the design problem. Although this section should be supported by a text discussion, it should be strongly supported by a detailed solid model of engineering analysis and design methods. Be sure to discuss the major subsystems in the design and the purpose and features of each subsystem. Include the following material to support this section:
%\begin{itemize}[noitemsep]
 %   \item A thorough presentation and discussion of all engineering analysis used in the design process. Present all formulations, assumptions and parameters used. Show results of the analysis.
  %  \item You should prove that the design will not fail and will perform as required through analysis. If you cannot predict it, then it is research or experiment, not engineered design.
   % \item Packaging Description (HW 4)
    %\item Hardware Design (HW 3)
    %\item PCB Design (HW 5)
    %\item Software Design (not a listing of code) (HW 7)
%\end{itemize}
\subsection{Design Constraints and Component Selection Rationale (HW 2)}

\subsubsection{Computation Requirements}

Computation requirements for the add-on PCB are not highly restrictive. The microcontroller must be powerful enough to run the USB protocol and facilitate communication between the drone’s onboard computer and the sensors connected to the USB hub.

\subsubsection{Interface Requirements}

The design requires several interfaces, including SPI, UART, and USB. To interface with the USB hub, the microcontroller must operate at 48\,MHz. The USB 2.0 specification also requires the board to supply 500\,mA of current at 5\,V.

There are three SPI devices on the board. While these could theoretically share a single SPI bus, the SD card—operating in SPI mode—necessitated the use of a dedicated SPI interface to simplify firmware development. The STM32 HAL driver for SPI was not optimized for this shared configuration. Additional pull-up resistors were added to each SD card data pin to enhance data integrity.

\subsubsection{On-Chip Peripheral Requirements}

The Harm Drone system requires multiple on-chip peripherals. These include:
\begin{itemize}
    \item Two SPI interfaces
    \item One UART interface
    \item One USB interface
    \item One PWM clock (for LED brightness control)
    \item A 48\,MHz clock source for USB hub communication
\end{itemize}

\subsubsection{Off-Chip Peripheral Requirements}

External components include:
\begin{itemize}
    \item USB hub (must interface with four external USB devices)
    \item SD card reader (must support SPI mode)
    \item SRAM and EEPROM (connected over SPI, used for future system expansions)
\end{itemize}

\subsubsection{Power Constraints}

Power is supplied by a 15\,V rechargeable battery that powers all subsystems. A two-stage regulation system includes:
\begin{itemize}
    \item A buck converter (15\,V $\rightarrow$ 5\,V for peripherals)
    \item An LDO regulator (5\,V $\rightarrow$ 3.3\,V for microcontroller, GPS, and sensors)
\end{itemize}

Brushless motors and ESCs draw directly from the 15\,V source, while lower-power components rely on precise regulation. The power system is optimized for:
\begin{itemize}
    \item 10 minutes of flight time
    \item Voltage stability
    \item Current distribution across subsystems
\end{itemize}

\subsubsection{Packaging Constraints}

Due to the Harm Drone’s compact, asymmetric design and protruding components (e.g., propeller arms, sensors, power connectors), the packaging must:
\begin{itemize}
    \item Be custom-fit to avoid damage
    \item Provide rigid outer casing with shock-absorbent interior
    \item Safely house the 15\,V battery and accessories
    \item Allow for easy unpacking, assembly, and re-deployment
\end{itemize}

\subsubsection{Cost Constraints}

The project operates under a \$325 budget, covering all hardware. To stay within this constraint:
\begin{itemize}
    \item Components were selected based on cost-performance tradeoffs
    \item Off-the-shelf parts were prioritized
    \item Critical parts (e.g., buck converter, USB hub) were chosen for affordability and efficiency
\end{itemize}

Despite budget limits, the design must remain competitive with commercial offerings, many of which benefit from economies of scale.

\subsection{Component Selection Rationale}

Component choices prioritized efficient power management, reliable communication, compact footprint, and low power consumption.

\begin{itemize}
    \item \textbf{Microcontroller:} STM32F072CB was selected for its ARM Cortex-M0 core, USB 2.0 support, UART/SPI interfaces, and flash/RAM balance~\cite{stm2023}. Table~\ref{tab:mcus} compares it with the ATmega32U4~\cite{atmel2023}.

    \item \textbf{USB-UART Converter:} FT231XS-R was chosen for stable USB-to-serial conversion~\cite{ftdi2023}. Table~\ref{tab:usb_uart} compares it with the CP2102~\cite{silabs2023}.

    \item \textbf{Buck Converter:} LMR51430XDDCR offers efficient 15\,V to 5\,V regulation~\cite{ti2023b}. Table~\ref{tab:buck} compares it with the MP2315~\cite{mp2315}.

    \item \textbf{USB Hub:} TUSB2046I provides 4-port USB expansion with low power consumption~\cite{ti2023a}. Compared in Table~\ref{tab:usb_hub} with the FE1.1S~\cite{fe1hub}.

    \item \textbf{LDO Regulator:} XC6210A332MR-G was selected for low dropout and high efficiency~\cite{torex2023}. Compared in Table~\ref{tab:ldo} with the AMS1117-3.3~\cite{ams1117}.

    \item \textbf{EEPROM:} CAT25160VI-GT3 supports reliable SPI-based data logging~\cite{onsemi2023}. Table~\ref{tab:eeprom} compares it with AT24C256~\cite{at24c256}.

    \item \textbf{SRAM:} IS62WVS2568GBLL-45NLI-TR provides high-speed memory access for real-time processing~\cite{issi2023}. Compared in Table~\ref{tab:sram} with 23LC1024~\cite{microchip2023}.
\end{itemize}

These components collectively form a robust and efficient embedded system that meets performance, power, and interface demands.

\subsection{Commercial Product Packaging}

To refine the \textit{Harm Drone}’s PCB packaging design, we analyzed two commercial drones with plastic-based enclosures: the Parrot AR.Drone and the Parrot Anafi USA. Both drones emphasize lightweight and durable construction, making them useful comparisons for our 3D-printed plastic case. However, they differ in structural integrity, modularity, and accessibility.

Parrot designed the AR.Drone with a plastic and expanded polypropylene (EPP) foam casing, which protects its electronics while keeping the drone lightweight and impact-resistant. Its interchangeable hulls include a foam indoor hull for propeller protection and a lightweight plastic outdoor hull for maneuverability \cite{parrot1}. While this design remains cost-effective and lightweight, it lacks structural rigidity and EMI shielding.

Parrot built the Anafi USA for professional and industrial applications, using lightweight thermoplastic materials to improve structural integrity and environmental resistance \cite{parrot2}. Unlike the AR.Drone, the Anafi USA features a reinforced plastic housing, which protects its internal flight electronics, sensors, and camera systems while maintaining modularity for easier maintenance. However, this model still prioritizes lightweight efficiency over impact resistance, making it more fragile than drones with metal enclosures.

\subsubsection{Parrot AR.Drone}

The Parrot AR.Drone serves as an early example of lightweight, plastic-based enclosures for drones. It encloses PCB components inside a light plastic shell, while internal EPP foam elements absorb impact. The enclosure integrates fully into the drone’s frame, ensuring efficient weight distribution and minor crash protection.

\textbf{Positive Aspects:}
\begin{itemize}
    \item Maintains lightweight and energy efficiency due to plastic and foam materials.
    \item Features interchangeable hulls for adaptability in different environments.
    \item Provides a cost-effective and easily manufacturable design.
\end{itemize}

\textbf{Negative Aspects:}
\begin{itemize}
    \item Offers limited structural protection due to its lightweight materials.
    \item Lacks dedicated EMI shielding, which may affect sensor performance.
    \item Uses a fixed casing, making maintenance and upgrades more difficult.
\end{itemize}

We adopted the lightweight plastic construction approach of the Parrot AR.Drone while improving structural reinforcement and accessibility. Instead of using a fully integrated casing like the AR.Drone, we designed a 3D-printed modular enclosure with two interlocking halves and a latch mechanism, allowing for quick access, easy maintenance, and upgrades. Additionally, we incorporated precisely positioned cutouts for USB-C, microSD, USB ports, XT60 connectors, and a power switch, enabling seamless external connectivity without requiring complete disassembly.

\subsubsection{Parrot Anafi USA}

Parrot engineered the Anafi USA to balance durability and weight efficiency, ensuring optimal flight performance while protecting internal electronics. Parrot built the drone’s enclosure as part of the main frame, ensuring structural support while keeping components compact and secure.

\textbf{Positive Aspects:}
\begin{itemize}
    \item Parrot used reinforced thermoplastic materials to improve durability while maintaining a lightweight design.
    \item The modular structure allows for easier maintenance and upgrades.
    \item The design includes basic EMI shielding, which reduces interference with sensor and communication systems.
\end{itemize}

\textbf{Negative Aspects:}
\begin{itemize}
    \item The lightweight plastic housing offers limited impact resistance compared to metal enclosures.
    \item Parrot designed the casing as non-removable, making internal components harder to access.
    \item The enclosure lacks vibration-dampening features, which could affect PCB stability over time.
\end{itemize}

We based the \textit{Harm Drone}’s enclosure on the Anafi USA’s thermoplastic construction, enhancing durability and accessibility through a custom 3D-printed modular design. Unlike the Anafi USA’s fully integrated frame, our enclosure maintains modularity, ensuring that components can be easily removed, replaced, or upgraded. To improve usability, we integrated cutouts for USB-C, microSD, USB ports, XT60 connectors, and a power switch, allowing external connections without the need for full enclosure removal.

\subsection{Project Package Specifications}

We designed the \textit{Harm Drone}’s PCB packaging as a lightweight, durable, and modular enclosure that protects critical electronic components while allowing easy maintenance and upgrades. The enclosure consists of a custom 3D-printed plastic case with two interlocking halves, secured using a latch mechanism. This structure enables quick access to internal components, eliminating the need for complex disassembly.

We selected 3D-printed plastic for its balance of strength and weight efficiency. The two-part housing connects via a latch, eliminating the need for screws or additional fasteners. Internal mounting points secure the PCB, reducing mechanical stress and vibration.

To support external connections, we incorporated custom cutouts for the power switch, USB-C port, microSD slot, USB ports, and XT60 connector. These cutouts align precisely with the PCB layout, ensuring seamless connectivity without requiring enclosure removal.

\subsection{PCB Footprint Layout}

We designed the \textit{Harm Drone}’s PCB footprint layout to optimize routing, space usage, and accessibility while aligning with the 3D-printed enclosure’s cutouts. We selected footprints based on space constraints, manufacturability, and electrical performance to ensure an efficient balance between density and accessibility.

The STM32F072CB microcontroller uses an LQFP-48 package, ensuring compact form and manageable pin spacing for UART, SPI, and USB routing \cite{stm32}. The FT231XS-R USB-to-UART converter and TUSB2046I USB hub use SOIC and QFN footprints, reducing PCB footprint while ensuring signal integrity \cite{ft231x} \cite{tusb2046i}. The CAT25160VI-GT3 EEPROM, placed in an 8-SOIC package, provides compact non-volatile storage \cite{cat25160}.

For high-speed memory access, we placed the IS62WVS2568GBLL SRAM in a TSOP-44 footprint, which balances pin density and thermal performance while allowing for compact placement on the board \cite{is62}. We positioned the MSD-12-A microSD card module, a critical component for external data storage, in an edge-mounted footprint to align directly with the enclosure’s cutout for user accessibility \cite{msd12}. We routed the 23LC1024 SPI bus, which the microcontroller, EEPROM, and SD card share, to minimize signal interference and optimize communication speeds \cite{23lc1024}.

We selected SOT-23-6 and SOT-23-5 footprints for power regulation components, including the LMR51430XDDCR buck converter and XC6210A332MR-G LDO regulator, to ensure efficient power conversion with minimal space usage \cite{lmr51430} \cite{xc6210}. We implemented through-hole mounting for high-stress connectors like USB and XT60 power, ensuring long-term mechanical stability.

With an estimated PCB size of 95.5\,mm $\times$ 40.228\,mm $\times$ 17.68\,mm, the layout maintains clear separation of power and data lines, ensures efficient signal routing, and aligns precisely with enclosure cutouts. We included the final footprint layout in Appendix C, detailing component placement, trace routing, and external port alignment.

\subsection{Performance Estimates and Results (2-5 pages)}
Present the estimated performance of the project (and how it was derived) based on the preliminary design (estimates to include speed, cost, power consumption, noise-immunity, ease of use, etc., depending on the project). Present the actual performance results. Discuss the results. Compare with estimated performance and explain discrepancies. Include suggestions for design changes that would improve the performance of the project. Use graphs or other figures to show relationships when appropriate.

%%%%%%%%%%%%%%%%%%%%%%%%%%%%%%%%%%%%%%%%%%%%%%%%%%%%%%%%%%%%%%%%%%%%%%%%%%
%                         Economic Analysis                              %
%%%%%%%%%%%%%%%%%%%%%%%%%%%%%%%%%%%%%%%%%%%%%%%%%%%%%%%%%%%%%%%%%%%%%%%%%%
\section{Economic Analysis}
\subsection{Cost Analysis}
Design includes some form of economic analysis. Realistic design must be concerned with cost. Include an analysis for:
\begin{itemize}[noitemsep]
    \item Prototype one-time cost (parts and implementation)
    \item Prototype labor investment (documented engineer-hours)
    \item Lifetime operational cost
\end{itemize}

\subsection{Bill of Materials}
Include a full parts list for the entire design if applicable. All standard parts should be completely identified by their code or specification. Custom parts must also be specified.

%%%%%%%%%%%%%%%%%%%%%%%%%%%%%%%%%%%%%%%%%%%%%%%%%%%%%%%%%%%%%%%%%%%%%%%%%%
%                    Reliability and Safety Analysis                     %
%%%%%%%%%%%%%%%%%%%%%%%%%%%%%%%%%%%%%%%%%%%%%%%%%%%%%%%%%%%%%%%%%%%%%%%%%%
\section{Reliability and Safety Analysis (HW 6)}
It is the ethical responsibility of the engineer to ensure that the solution to the design problem is safe to the public and the environment. This is substantiated by showing that Design for Safety methods were employed in the design process and documented through a Hazards and Failure Analysis. Discuss the results of the analysis and how safety was incorporated into the design.

%%%%%%%%%%%%%%%%%%%%%%%%%%%%%%%%%%%%%%%%%%%%%%%%%%%%%%%%%%%%%%%%%%%%%%%%%%
%               Social/Political/Environmental Impact                    %
%%%%%%%%%%%%%%%%%%%%%%%%%%%%%%%%%%%%%%%%%%%%%%%%%%%%%%%%%%%%%%%%%%%%%%%%%%
\section{Social/Political/Environmental Impact (HW 8)}

We designed the Harm Drone PCB with a strong emphasis on ethical engineering and social responsibility. Because the board collects real-time data in potentially sensitive environments, we intentionally avoid wireless data transmission, reducing interception risks and aligning with ethical data handling practices~\cite{parrot2023,microchip2023}.

We developed rigorous testing protocols to confirm safe operation under conditions such as fluctuating power, temperature changes, and data load variations. By incorporating fail-safe mechanisms, reliable firmware, and complete documentation, we reduce the likelihood of failure during critical missions.

Our documentation includes all necessary electrical, integration, and operational details to ensure safe and transparent deployment. This level of openness aligns with the National Society of Professional Engineers (NSPE) Code of Ethics, which emphasizes engineers' responsibility to protect public safety and communicate system limitations~\cite{nspe}.

\subsection{Stakeholder Considerations}

The system directly impacts multiple stakeholders, including Professor Harms' lab, drone operators, academic researchers, and communities near drone operation zones. We ensure secure data handling by enforcing local storage and by not implementing wireless communication, which reduces the chance of unintended surveillance. This design approach strengthens trust between developers, users, and the public, and supports transparent research practices~\cite{parrot2023}.

\subsection{Research Data Privacy and Ethics}

Because the board may store geolocation and environmental data from sensitive areas, we prioritize data privacy in our system architecture. By avoiding wireless data transfers and encrypting log files where necessary, we ensure compliance with data privacy standards such as the California Consumer Privacy Act (CCPA)~\cite{microchip2023}. Researchers integrating the board must follow institutional and governmental review processes for ethical deployment.

\subsection{Political Impact Analysis}

Although the Harm Drone PCB is designed for academic use, its integration of USB interfaces, microcontrollers, and memory modules introduces political and regulatory complexities. These components may fall under U.S. export control regulations such as ITAR or EAR when paired with sensitive payloads~\cite{itar}.

We source parts from vendors who comply with international trade regulations and maintain transparency in their labor practices. Should the board see international deployment, we will ensure compliance with American data protection standards such as CCPA and with product safety regulations like CE and FCC~\cite{ti2023b,microchip2023}.

To mitigate supply chain risks, we identified multiple vendors and ensured compatibility with alternative components. Our adherence to RoHS, CE, and FCC standards promotes safe design and supports potential international academic collaboration~\cite{ti2023b}.

\subsubsection{Regulatory Compliance}

We ensured compliance with globally recognized standards, including RoHS (for hazardous substance restrictions), FCC (for electromagnetic emissions), and CE (for general product safety). Meeting these standards strengthens product safety and legal clarity, especially for future research use or potential commercial adaptation~\cite{ti2023b}.

\subsection{Environmental Impact Analysis}

Although we did not manufacture an entire drone platform, the PCB system still contributes environmental impacts throughout its lifecycle. We addressed these challenges through design decisions that reduce waste, power consumption, and end-of-life hazards.

\subsubsection{Manufacturing}

PCB fabrication consumes energy and involves hazardous chemicals such as etchants and solvents. To minimize this impact, we use RoHS-compliant materials, lead-free solder, and compact surface-mount designs that reduce board size and component count~\cite{ti2023b}. We followed IPC-2221B design standards to ensure environmentally responsible layout practices~\cite{ipc2221b}.

\subsubsection{Enclosure Material and Additive Manufacturing}

To house and protect the PCB during deployment, we designed a custom 3D-printed enclosure made from polylactic acid (PLA), a biodegradable thermoplastic derived from renewable resources like cornstarch or sugarcane. PLA offers several benefits over petroleum-based plastics, including a lower carbon footprint during production and improved biodegradability under industrial composting conditions~\cite{pla2023}.

By using additive manufacturing, we minimize material waste and enable rapid iteration of enclosure designs to fit a variety of drone platforms. This approach supports our modular integration goals and allows users to fabricate new casings on demand using common 3D printing equipment. The use of PLA also avoids the environmental hazards associated with legacy enclosures made from ABS or carbon-reinforced polymers.

\subsubsection{Normal Use}

During operation, the PCB draws low power from the drone’s main supply. We optimized firmware to reduce polling cycles and power consumption from non-critical modules. Our modular design supports repairability, allowing users to replace only faulty components—such as USB headers or memory chips—rather than the entire board.

\subsubsection{Disposal/Recycling}

At end-of-life, the board presents moderate recycling challenges due to embedded copper, tin, and trace amounts of gold. To mitigate these risks, we included clear disassembly documentation and recommended disposal via certified e-waste programs. Our packaging uses recyclable materials and avoids excessive padding, aligning with environmentally responsible delivery methods~\cite{epa2023}.

\subsection{Lifecycle Assessment Overview}

By decoupling the data logging system from specific drone models, we reduce redundant hardware production and lower our carbon footprint. The board’s reusability across multiple UAV platforms increases its service life and reduces overall resource consumption. Our design choices support circular lifecycle goals through modularity, durability, and sustainable manufacturing practices~\cite{epa2023}.
%%%%%%%%%%%%%%%%%%%%%%%%%%%%%%%%%%%%%%%%%%%%%%%%%%%%%%%%%%%%%%%%%%%%%%%%%%
%            Discussion, Conclusions, and Recommendations                %
%%%%%%%%%%%%%%%%%%%%%%%%%%%%%%%%%%%%%%%%%%%%%%%%%%%%%%%%%%%%%%%%%%%%%%%%%%
\section{Discussion, Conclusions, and Recommendations (2-4 pages)}
Restate the problem that gave rise to the report. Summarize the main points and the approach that was taken. Summarize the design performance. Provide recommendations, explaining subsequent action or posing specific questions for investigations. Discuss the lessons learned.

%%%%%%%%%%%%%%%%%%%%%%%%%%%%%%%%%%%%%%%%%%%%%%%%%%%%%%%%%%%%%%%%%%%%%%%%%%
%                         User’s Manual                                 %
%%%%%%%%%%%%%%%%%%%%%%%%%%%%%%%%%%%%%%%%%%%%%%%%%%%%%%%%%%%%%%%%%%%%%%%%%%
\section{User’s Manual (1-3 pages)}
Provide a user’s manual for the operation and maintenance of the system designed in the project. The operators' manual for a Senior Capstone Design Project will give complete and detailed step-by-step operating instructions such that anyone can operate the device. If it is necessary to calibrate any part of the project to assure its proper operation, the calibration procedure must be a part of this manual. It will contain all applicable safety warnings that apply to operation and/or operator maintenance of the project.

The user's manual will include all initialization and/or set up instructions needed for the proper operation of the project. If the project is done for yourself, the manual may be bound in the report and need only be detailed enough that you would be able to operate it after a long period of time away from it.

%%%%%%%%%%%%%%%%%%%%%%%%%%%%%%%%%%%%%%%%%%%%%%%%%%%%%%%%%%%%%%%%%%%%%%%%%%
%                          Appendices                                    %
%%%%%%%%%%%%%%%%%%%%%%%%%%%%%%%%%%%%%%%%%%%%%%%%%%%%%%%%%%%%%%%%%%%%%%%%%%
\section{Appendices}
Include in the appendices information that could not be included in the formal body of the report because it would disrupt the continuity of the discussion. Background materials, product catalogs, experimental data tables, and extra documentation should be placed in the appendix.

The following appendices should be organized in the order listed. If an appendix is not applicable to your report, insert a page in the proper place and note this fact on it.

\subsection*{Appendix A: Notes}
This appendix consists of a numbered list of material, books, magazines, etc. from which you have taken information, circuits, data, or other information directly. The main body of the report is to use End Notes for references. Material in the main body of the report taken from any of the listed documents should be assigned an End Note number as a superscript in the discussion of the material and be listed in the END note section. In the case of schematics and pictures, the end note number should appear as a superscript after the figure's title. The first time a copyright or trademark is encountered in the report it
will be assigned an end note number and proper notation will appear in the end notes. Written permission will be sought for information copied from data books, books, magazines, and data sheets. If verbal permission is given, the date, name, and phone number of the person giving permission will be noted in the end notes. If permission cannot be obtained, no copies of the data will be included in the report.

\subsection*{Appendix B: Acceptance Testing}
This appendix is an extract from your senior project proposal requirements specifications testing section. If you have requested a change in your acceptance testing (Project Specific Success Criteria) a copy of the request and the approval letter will be included. The last page will then be a copy of the new (modified) acceptance testing criteria (PSSC). This information will be used by the faculty to judge whether your project performs as proposed or has been modified during the demonstration phase of your presentation.

\subsection*{Appendix C: Electrical Specifications}
This appendix will, as a minimal requirement, contain the following:
\begin{enumerate}[noitemsep]
    \item Schematics: An overall block diagram or interconnection diagram from ORCAD will be the first page. Lines on the schematics that enter and leave the page will be annotated so that one knows which page sheet they go to or come from as well as the name of the function it performs. Signal flow if possible, should be from left to right. If a line is exceptionally long its name should appear above the line about midway of the total length. Devices will be assigned reference designations as listed in the national standard (IEEE Std 315-1975). If a reference designation is not found in the quick reference list for a device, further information may be obtained from the standards books maintained in the ECEN maintenance shop. Symbols drawn by the ORCAD software will also be acceptable. Logic symbols will be in accordance with IEEE Std 91-1984 or as drawn by ORCAD.
    \item Timing Analysis: A timing analysis will be done on all devices necessary.
    \item Loading Analysis: A loading analysis will be done on all outputs necessary.
    \item Specification Sheets: Only include data sheets of unique or seldom used parts. Check with the manufactures to gain permission to reproduce data sheets for inclusion in your report.
    \item Signal Quality Analysis (SQA): An SQA analysis will be done on all signals transmitted on long runs.
    \item Safety/Electrical Hazard Checklist: With respect to safety, all controls will be appropriately labeled. Warning labels such as “disconnect power before opening case”, “no user serviceable parts inside”, “danger hazardous voltages” etc. will be affixed to the project if applicable.
    \item Accuracy Certification: In this section, describe how the accuracy claimed in your project acceptance testing statement (PSSC) was verified. Give the name and serial number of the units used to check the accuracy and the date of the most recent time the instrument was calibrated against a NIST traceable standard.
\end{enumerate}

\subsection*{Appendix D: Software}
\begin{enumerate}[noitemsep]
    \item Flowcharts: An overall flowchart will begin this section. Detailed flowcharts will be done for each individual routine. Each flowchart will be given a descriptive title. The title should be centered at the top of the page. On the line below the title, centered and enclosed in parenthesis, will appear a reference to the corresponding line numbers or page in the software listing where this code can be found. Circles containing an alpha designator will be used to show connectivity between portions of flowcharts. A to/from page number will be placed next to the circle to aid in tracing the flow between pages. Pseudo-code may be used in place of flowcharts for computational code, but not for control or input/output code.
    \item Program Listings: High Level or Machine: Compressed printing can be used. All programs must have enough comments so that a reviewer can gain insight into the purpose of the program elements.
\end{enumerate}

\subsection*{Appendix E: Resource Expenditure Analysis}
\begin{enumerate}[noitemsep]
    \item Cost Analysis: This appendix consists of a list of parts grouped under general headings along with the total price. In narrative form, discuss the nature of any cost overrun, if you are more than 10\% over your estimated cost.
    \item Labor Hour Analysis: This appendix will contain a breakdown of the hours spent in the development of the hardware and the software. Besides these two main areas, attempt to sub divide it into specific areas as basic research, requisition of parts, testing hardware, testing software, debugging hardware/software, writing the report, etc.
    \item Parts List: This appendix will contain a listing, in reference designation alpha numeric order, of all the parts used. All parts which do not have a reference designation will be listed in alpha numeric order under a heading of miscellaneous items. Use of the parts list documentation of the schematic capture part of the ORCAD program is recommended.
    \item Other Resources: Include in this appendix things like funding supplied by others, review time done by others, etc.
\end{enumerate}

\subsection*{Appendix F: Individual Student Outcomes Appendices}
Each member of the team is responsible for an individual appendix (F1, F2, F3, [F4]) that includes evidence that each member of the team has met ABET Criterion 3, 1) through 7), student outcomes. These can be organized as you wish, but they MUST BE ORGANIZED, not just a collection of notes and miscellaneous course work. Identify each of the 7 outcomes and your supporting evidence.

%%%%%%%%%%%%%%%%%%%%%%%%%%%%%%%%%%%%%%%%%%%%%%%%%%%%%%%%%%%%%%%%%%%%%%%%%%
%                    Report General Guidelines                           %
%%%%%%%%%%%%%%%%%%%%%%%%%%%%%%%%%%%%%%%%%%%%%%%%%%%%%%%%%%%%%%%%%%%%%%%%%%
\section*{Report General Guidelines}
\begin{itemize}[noitemsep]
    \item Be sure to introduce and summarize each section.
    \item Always write general to specific in each section.
    \item Do not write chronologically. (a technical report is not a story or a novel)
    \item Use section and subsection titles.
    \item Make sure that subsections follow each other in a logical progression.
    \item Number each page.
    \item Use bulleted or enumerated lists rather than lengthy textual discussion of requirements, subsystems, etc.
\end{itemize}

\section*{Figures and Tables}
\begin{itemize}[noitemsep]
    \item Technical reports only contain Figures and Tables.
    \item Refer to graphs as figures, photos as figures, small code segments as figures, etc.
    \item Figures and tables are NOT to be hand sketched.
    \item Figures and tables should be used to supplement the discussion.
    \item Always introduce a figure or table in the text and never place a figure or table in the text that is not discussed.
    \item Discuss the meaning and significance of the table or figure.
    \item Be sure to highlight the fine points and structure.
    \item Figures and tables should be located in the body of the text, AFTER they are introduced in the text.
    \item It is often appropriate to pull out small segments of code from a main program or to write pseudo-code to describe an algorithm or major point of the project. This is considered a figure and should be titled and numbered as such.
    \item If a group of figures or a long table or code listing takes up too much space, locate them in an appendix.
    \item Figures and tables can be located at the end of the text, but it is less convenient for the reader.
    \item Figure titles and numbering: Figures should be numbered consecutively in the report. Every figure must have a descriptive title located immediately below the figure.
    \item Table titles and numbers: Tables should be numbered consecutively in the report. Every table must have a descriptive title located above or below the table.
\end{itemize}

\section*{References}
Use the IEEE format for reference style. List all references used in the report. All references should include author, title, journal or magazine title (if a journal article), publisher, page number, date. Below are sample references from a conference proceeding paper [1], book [2], journal article [3], Ph.D. dissertation [4], technical specification [5], and web page.

\begin{enumerate}[noitemsep]
    \item [1] P. J. Hurst and W. J. McIntyre, “Double sampling in switched-capacitor delta-sigma A/D converters,” in Proc. IEEE Int. Symposium on. Circuits and Systems., 1990, pp. 902-905.
    \item [2] J. C. Candy and G. C. Temes, \textit{Oversampling Delta-Sigma Data Converters: Theory, Design and Simulation}. New York: IEEE Press, 1992.
    \item [3] L. R. Rabiner, R. W. Schafer, and C. M. Rader, “The chirp z-transform algorithm,” \textit{IEEE Trans. on Audio Electroacoustics}, AU-17:6 (June 1969), pp. 86-92.
    \item [4] S. Bagchi, “The nonuniform discrete Fourier transform and its applications in signal processing,” Ph.D. dissertation, Electrical Engineering Department, Univ. California, Santa Barbara, 1994.
    \item [5] Motorola CMOS Logic Data, Series C, Motorola, INC, 1990, pp. 6-97 - 6-107.
    \item [6] EE Design Center - Questlink Technology, www.questlink.com, 1999.
\end{enumerate}

%%%%%%%%%%%%%%%%%%%%%%%%%%%%%%%%%%%%%%%%%%%%%%%%%%%%%%%%%%%%%%%%%%%%%%%%%%
%                              End Document                              %
%%%%%%%%%%%%%%%%%%%%%%%%%%%%%%%%%%%%%%%%%%%%%%%%%%%%%%%%%%%%%%%%%%%%%%%%%%
\end{document}
