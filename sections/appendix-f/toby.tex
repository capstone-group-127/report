\documentclass[../../main.tex]{subfiles}

\begin{document}

\subsubsection{Appendix F1: Toby Heinemann}

% an ability to identify, formulate, and solve complex engineering problems by applying principles of engineering, science, and mathematics.
\paragraph{Principles}
\par Throughout the report, it has been shown that Toby was able to identify, formulate, and solve complex engineering problems. Toby assisted in identifying the needs of the lab, formulating a problem based on these needs, and developing a solution to this problem. As the resource manager, he led the effort to draft a formal definition of the problem that was later approved by the project sponsor and senior project officer (SPO). He then took control of the effort to plan out a timeline on how to solve the problem, as well as how to deligate responsibilities and resources. 

% an ability to apply engineering design to produce solutions that meet specified needs with consideration of public health, safety, and welfare, as well as global, cultural, social, environmental, and economic factors.
\paragraph{Design}
\par As stated in the report, Toby contributed to the overall design of the solution after the problem was identified and defined. He worked with his team to design the overall block diagram and system architecture for the project while keeping in mind the needs of the end user. He also worked to carefully design specifications that specifically adhered to FAA and IEEE standards.These specifications also kept the health and safety of the user in mind. For example, Toby specifically picked out the IEEE standard that would be adhered to, which dealt with the flicker frequency for the lighting system of the project. This was specifically chosen with the user in mind, and a safe flicker frequency for this system was chosen. Toby also worked to design individual hardware and software components of the project as well. He was able to successfully design, test, and implement the buck converter that was used to step down the power from the drone battery. He also assisted in programming the SPI bus to be used with the EEPROM and SRAM.

% an ability to communicate effectively with a range of audiences.
\paragraph{Communication}
\par As the resource manager, Toby was in charge of handling the communicaiton between the team, the project sponsor, and the SPO. He was in charge of scheduling times during the week and on the weekends to meet, as well as scheduling times to meet with the SPO and project sponsor. Furthermore, he was a part of a final presentation with his team. During this presentation, he was responsible for communicating to the audience various aspects of how the project came about and how the problem was defined. He also explained how the lighting system was designed and tested.

% an ability to recognize ethical and professional responsibilities in engineering situations and make informed judgments, which must consider the impact of engineering solutions in global, economic, environmental, and societal contexts.
\paragraph{Professionalism}
\par Throughout the course, Toby considered his ethical and professional responsibilities. While the design was being constructed and revised, Toby worked to research various patents, and reached out to the relavant people to make sure nothing would be infringed upon with the design. Furthermore, a survey on an ethical dillema was given during the lecture, and Toby provided valid responses on how he would manage the situation in an ethical way.


% an ability to function effectively on a team whose members together provide leadership, create a collaborative and inclusive environment, establish goals, plan tasks, and meet objectives.
\paragraph{Teamwork}
\par Toby worked effectively on a team during the duration of the whole project. From the very beginning, he helped to formulate a project charter and establish performance metrics for the team. He then assisted in the development of the project plan, where the project schedule and responsibilities were defined. As the resource manager, he was also in charge of scheduling weekly team meetings, as well as meetings with the SPO and project sponsor. He also took part in weekly accountibility meetings, or WAMs, in which the deliverables from the previous week, the risks, and deliverables for the next week were stated. 

% an ability to develop and conduct appropriate experimentation, analyze and interpret data, and use engineering judgment to draw conclusions.
\paragraph{Analysis}
\par To complete the project, 5 projcet specific success criteria, or PSSCs, had to be defined. These were certain features of the project that had to be demonstrated in order to prove to the spoonsor and the SPO that it was successful. Toby assisted in the testing of each of these, and worked to interpred and analyze the results. Furthermore, when parts of the project were not working as expected, Toby assisted in the analysis and debugging to find the root cause. This involved using devices such as a multimeter, oscilloscope, and logic analyzer to read and intrepret data and electrical signals. 

% an ability to acquire and apply new knowledge as needed, using appropriate learning strategies.
\paragraph{Learning}
\par Toby demonstrated that he was able to learn new skills and apply knoweldge as needed throughout the duration of the project. Toby was able to successfully design and implement a buck converter, which was not something that was covered in previous coursework. Furhtermore, Toby was able to work with his teamates to learn more about different communication protocols that he did not have prior experience working with, such as SPI, USB, and UART. Toby not only learned about the hardware interfaces for these protocols, but also about the programming and timing.

\end{document}
