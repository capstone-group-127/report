\documentclass[../main.tex]{subfiles}

\begin{document}

\section*{Project Abstract}

\par The \textit{Harm Drone} project presents a low-cost, modular embedded system for unmanned aerial vehicles (UAVs) to log high-fidelity sensor data without relying on wireless telemetry or proprietary interfaces. Developed as a senior capstone at the University of Nebraska, the system emphasizes affordability, platform independence, and safety compliance.

\par The team designed a custom printed circuit board (PCB) featuring an STMicroelectronics microcontroller, Universal Serial Bus (USB) hub, with both volatile and nonvolatile onboard storage. The system also supports exporting to an SD card, enabling redundant data logging across Windows, macOS, and Linux systems. Compliance with IEEE 1789-2015 lighting standards ensures safe operation in public environments.

\par The Harm Drone comprehensively addressed the entire engineering lifecycle, beginning with problem definition and extending through PCB layout to firmware development and system validation. The scope included hardware prototyping, data integrity testing, packaging design, and regulatory assessment. Testing verified the system's ability to support up to four sensors, provide 1000 kilobits of nonvolatile storage, facilitate 4 gigabits of SD card logging, and ensure full compatibility with Windows, macOS, and Linux systems.

\par The system is fully prepared for UAV integration. Future improvements include support for more advanced sensors and environmental hardening. The \textit{Harm Drone} proves that robust, research-grade systems can be built affordably.

\end{document}
