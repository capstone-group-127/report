\documentclass[12pt]{article}
\usepackage[utf8]{inputenc}
\usepackage[tmargin=1in,lmargin=1in,rmargin=1in]{geometry}
\usepackage{fancyhdr}
\setcounter{tocdepth}{3}
\usepackage{titletoc,tocloft}
\setlength{\cftsubsecindent}{0.65cm}
\setlength{\cftsubsubsecindent}{1.3cm}
\newcommand{\nocontentsline}[3]{}
\let\origcontentsline\addcontentsline
\newcommand\stoptoc{\let\addcontentsline\nocontentsline}
\newcommand\resumetoc{\let\addcontentsline\origcontentsline}
\makeatletter
\newcommand\teenyparagraph[1]{%
  \@tempswatrue
  \if@nobreak\@tempswafalse\fi
  \if@noskipsec \leavevmode \fi
  \par
  \@nobreakfalse
  \everypar{}%
  \protected@edef\@svsec{\@seccntformat{paragraph}}%
  \@tempswafalse
  \@xsect{3.25ex \@plus 1ex \@minus .2ex}%
  {\normalfont\normalsize\bfseries #1}%
}
\makeatother
\usepackage{multicol}
\usepackage{setspace}
\usepackage{graphicx} % Required for inserting images
\usepackage{etoolbox}
\usepackage{enumitem}
\setlist[itemize]{noitemsep}
\usepackage{listings} % For code listings
\usepackage{color} % For colored text
\usepackage{amsmath} % For mathematical notation
\usepackage{array} % For better table formatting
\usepackage{float}
\usepackage{url}
\patchcmd{\thebibliography}{\section*}{\section}{}{}
\graphicspath{ {./images/} }

\newcommand{\quickimage}[2]{%
\begin{figure}[!htbp]
\centering
\includegraphics[width=#2]{#1}
\end{figure}%
} 

\newcommand{\quickfigure}[4]{%
\begin{figure}[!htbp]
\centering
\includegraphics[width=#2]{#1}
\caption{#3}
\label{#4}
\end{figure}%
} 

\lstset{ %
  numbers=left,           
  stepnumber=1,           
  numbersep=5pt,          
  showspaces=false,           
  showstringspaces=false,     
  showtabs=false,         
  frame=single,           
  tabsize=4,              
  captionpos=b,           
  breaklines=true,            
  breakatwhitespace=false,        
}
\begin{document}

\section*{Abstract}

\par The \textit{Harm Drone} project delivers a low-cost, modular embedded system that enables UAVs to log high-fidelity sensor data without relying on wireless telemetry or proprietary interfaces. Developed as a senior capstone within the University of Nebraska’s Electrical and Computer Engineering department, the purpose of this project was to design a platform-independent data logging system that emphasizes affordability, reliability, and safety compliance.

\par The team employed an iterative engineering design process, integrating an STM32 microcontroller, USB hub, EEPROM, and dual memory storage architecture into a custom PCB. Sensors connected via USB interfaces log data to onboard flash and a removable SD card, ensuring cross-platform accessibility. Key safety standards such as IEEE 1789-2015 lighting compliance were also incorporated.

\par Spanning the full engineering lifecycle. Starting with the problem definition and PCB layout to firmware development and system validation the scope of this project included hardware prototyping, data integrity testing, packaging design, and regulatory assessment. Testing confirmed support for up to four sensors, 1000KB of nonvolatile storage, 4GB of SD card logging, and full compatibility with Windows, macOS, and Linux systems.

\par Results show that the \textit{Harm Drone} system performs reliably in UAV conditions while remaining cost-effective and adaptable. Future recommendations include adding optional wireless modules, improving environmental sealing, and expanding firmware compatibility. The project demonstrates that performance and usability need not be sacrificed to achieve affordability making the \textit{Harm Drone} an ideal platform for research, education, and field applications.

\end{document}
